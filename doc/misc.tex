
\documentclass{article}

\usepackage{nopageno}
\usepackage{vmargin}
\usepackage{comment}
\usepackage{xspace}
\usepackage[hyperfootnotes=false]{hyperref}
\usepackage{mdframed}
\usepackage{xcolor}

\definecolor{lightblue}{rgb}{0.8,0.8,1.0}

\usepackage{../slides/hurd}
%\usepackage{tikz}
\usepackage{tkz-graph}
\usetikzlibrary{positioning, fit, backgrounds, arrows, trees, shapes, arrows.meta}

% I like smaller margins than the default
\setmargnohfrb{0.5in}{0.5in}{0.5in}{0.5in}

% I prefer unindented paragraphs with a blank line between them
\parindent 0pt
\parskip \baselineskip

% Redefine underscore to print as a normal character
\catcode`\_=\active
\def_{\_}

\title{Miscellaneous Notes on GNU/Hurd}
\author{Brent Baccala}
\date{May 2019}

\pdfinfo{
  /Title	(Miscellaneous Notes on GNU/Hurd)
  /Author	(Brent Baccala)
}

% Shortcuts just for this document
\def\libpager{{\tt libpager}\xspace}
\def\rpctrace{{\tt rpctrace}\xspace}
\def\netmsg{{\tt netmsg}\xspace}
\def\pagersync{{\tt pager_sync()}\xspace}
\def\pagerwritepage{{\tt pager_write_page()}\xspace}
\def\pagerreadpage{{\tt pager_read_page()}\xspace}
\def\moinit{{\tt memory_object_init}\xspace}
\def\moready{{\tt memory_object_ready}\xspace}
\def\morequest{{\tt memory_object_data_request}\xspace}
\def\mosupply{{\tt memory_object_data_supply}\xspace}
\def\modataerror{{\tt memory_object_data_error}\xspace}

%% Disable hypertext links in table of contents
%% https://tex.stackexchange.com/questions/55190

\makeatletter
\let\Hy@linktoc\Hy@linktoc@none
\makeatother

\begin{document}

\maketitle

\tableofcontents

\vfill\eject

\section{Blocking Behavior of {\tt mach_msg()}}

An issue with Mach's design caused me no little bit of grief.
Mach's {\tt mach_msg()} (as I read the documentation),
allows for non-blocking sends: according to \cite{interfaces}:

\begin{quote}
MACH_SEND_TIMEOUT

The {\it timeout} argument should specify a maximum time (in
milliseconds) for the call to block before giving up. If the message
can’t be queued before the timeout interval elapses, then the call
returns MACH_SEND_TIMED_OUT. A zero timeout is legitimate.
\end{quote}

In fact, a call to {\tt mach_msg()} can block indefinitely, even if a
{\it timeout} value of zero is specified!  The problem occurs when the
message is an RPC directed at the kernel which in turns triggers other
RPCs.  The memory manager is a big offender here.  Sending a {\tt
  vm_map} message to the kernel will cause the kernel to send a
\moinit message to the memory manager, then block waiting for a reply.
If both the {\tt vm_map} and the \moinit were handled by a single
thread in \netmsg, then that thread would block indefinitely,
requiring considerable additional complexity in \netmsg to ensure that
additional threads are always available in case calls to {\tt
  mach_msg()} block.

The consensus on the Hurd mailing list leaned in favor of this being
legitimate behavior, reading the documentation as guaranteeing that
the timeout applied to the send, and not to the underlying RPC.  In
any event, changing this behavior would require quite a bit of work on
the Mach kernel.

\section{\rpctrace}

\rpctrace is Hurd's version of {\tt strace}.  It works by wrapping
around all of a process's ports, intercepting all Mach messages send
to and from that process, and printing them out.

It's got some problems, starting with the difficulty of attaching
and detaching running processes.

Attaching -- if the process has a receive right, we want to move the
receive right to \rpctrace (all of the send rights remain attached to
the same queue) and insert a new receive right in the process.  If one
of the process's threads is blocked trying to receive a message on
that receive right at the moment of \rpctrace attachment, the {\tt
  mach_msg} call will return with an {\tt MACH_RCV_PORT_DIED} error.
Portsets don't cause this kind of problem, since we can move receive
rights in and out of them without triggering errors returns from
{\tt mach_msg}.

Detaching -- there's a race condition here.  Could use a new Mach
primitive to combine two queues together.


\section{New Mach System Calls}

\begin{itemize}

\item {\tt mach_port_merge_ports}

Given a send right and a receive right to two different ports, collapse
them together so that the receive right and everything that was in its
queue goes to the send right's port, as if all messages on the receive
right's port were atomically removed and sent to the send right.

\item {\tt mach_port_split_port}

Given a receive right, split it, returning a new receive right such
that all of the old port's send rights now point to the new receive
right, but any {\tt mach_msg()} calls that were in progress on the
receive right continue un-interrupted.  Messages in the old port's
queue go to the new receive right.  In short, we just move the old
port/queue to a new name, but continue any {\tt mach_msg()} calls
uninterrupted.

\item {\tt thread_get_state} and {\tt thread_set_state}

How to handle thread state when the process is in a system call?

We want to be able to save state well enough to re-create the
thread, maybe on a different node, and restart from the point
it was suspended.

Just re-running the system call would result in messages being
transmitted twice.

\end{itemize}


\begin{thebibliography}{9}

\bibitem{interfaces} {\it Mach 3 Kernel Interfaces}.

\bibitem{principles} {\it Mach 3 Kernel Principles}.

\end{thebibliography}


\end{document}
