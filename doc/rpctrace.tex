\documentclass{beamer}
\usetheme{Madrid}

\setbeamertemplate{footline}{}
\beamertemplatenavigationsymbolsempty

\usepackage{comment}
\usepackage{hurd}

\usetikzlibrary{positioning, calc, arrows, fit}

\begin{comment}

;; ELISP code to insert and remove slides from beamer \only macros

(defun slide-func (slide)
  "add or remove a slide number from only macros"
  (let ((mystr (match-string 0)) (start 0) value1 value2 newstr (sign (/ slide (abs slide))))
       (save-match-data (while (string-match "\\([0-9]+\\)\\(-\\([0-9]+\\)?\\)?" mystr start)
                               (setq value1 (string-to-number (match-string 1 mystr)))
                               (setq value2 (cond ((match-string 3 mystr) (string-to-number (match-string 3 mystr)))
                                                  ((match-string 2 mystr) 1000000)
                                                  (t value1)))
                               (if (> value1 (abs slide)) (setq value1 (+ value1 sign)))
                               (if (>= value2 (abs slide)) (setq value2 (+ value2 sign)))
                               (setq newstr (cond ((> value1 value2) "0")
						  ((eq value1 value2) (format "%d" value1))
                                                  ((>= value2 999999) (format "%d-" value1))
                                                  (t (format "%d-%d" value1 value2))))
                               (setq mystr (replace-match newstr nil nil mystr))
                               (setq start (+ (match-beginning 0) (length newstr)))))
       mystr))

(defun slide-duplicate (slide)
  "remove slide numbers in only macros"
  (interactive "NSlide number to duplicate: ")
  (query-replace-regexp "\\only<[0-9-,]+>" `(replace-eval-replacement replace-quote (slide-func ,slide))))

(defun slide-remove (slide)
  "remove slide numbers from only macros"
  (interactive "NSlide number to remove: ")
  (query-replace-regexp "\\only<[0-9-,]+>" `(replace-eval-replacement replace-quote (slide-func ,(- slide)))))

\end{comment}

\begin{document}

\tikzstyle{send} = [rectangle, fill=green, rounded corners]
\tikzstyle{recv} = [rectangle, fill=red!50, rounded corners]
\tikzstyle{empty} = [minimum size=0, inner sep=0]
\tikzstyle{queue} = [draw, rectangle, minimum height=.5cm, minimum width=1cm, pattern=vertical lines]
\tikzstyle{message} = [draw, fill=red!20, text width=5em, text depth=3em, text centered,
                       minimum width=5em, minimum height=2em, rounded corners]

\tikzstyle{port link} = [black, -triangle 90]
\tikzstyle{message path} = [violet, thick, -triangle 90]
\tikzstyle{right link} = [red, thick, dotted, ->]
\tikzstyle{internal link} = [dashed, thick]

\begin{frame}
\begin{tikzpicture}

\useasboundingbox (-6,-5) rectangle (6.5,5.5);
\scope[transform canvas={scale=0.8}]

\task {rpctrace}{}{blr}{}

\task [above right = of rpctrace.north east] {Traced Task 1}{}{sl}{}
\task [below = of Traced Task 1] {Traced Task 2}{}{sl}{}
\task [below = of Traced Task 2] {Traced Task 3}{}{sl}{}
\task [below = of Traced Task 3] {Traced Task 4}{}{sl}{}

\task [left = of rpctrace] {Unknown Task}{}{r}{
}

\node [message] at (Unknown Task) {};

\draw [port link] (rpctrace Left Port 0) -- (Unknown Task Port 7);
\draw [port link] (Traced Task 1 Port 0) -| ($(Traced Task 1 Port 0)!.6!(rpctrace Right Port 5)$) |- (rpctrace Right Port 5);
\draw [port link] (Traced Task 2 Port 0) -| ($(Traced Task 2 Port 0)!.4!(rpctrace Right Port 0)$) |- (rpctrace Right Port 0);
\draw [port link] (Traced Task 3 Port 0) -| ($(Traced Task 3 Port 0)!.5!(rpctrace Right Port -5)$) |- (rpctrace Right Port -5);

\node (Task 1 sendright) at (Traced Task 1 Port 0) [send, anchor=west] {S};
\node (Task 2 sendright) at (Traced Task 2 Port 0) [send, anchor=west] {S};
\node (Task 3 sendright) at (Traced Task 3 Port 0) [send, anchor=west] {S};

\node (Task 4 recvright) at (Traced Task 4 Port 0) [recv, anchor=west] {R};

\node (Unknown Task recvright) at (Unknown Task Port 7) [recv, anchor=east] {R};
\node (Unknown Task sendright) at (Unknown Task Port -6) [send, anchor=east] {S};

\node (R1) at (rpctrace Right Port 5) [recv, anchor=east] {R};
\node (R2) at (rpctrace Right Port 0) [recv, anchor=east] {R};
\node (R3) at (rpctrace Right Port -5) [recv, anchor=east] {R};

\node (S1) at (rpctrace Left Port 0) [send, anchor=west] {S};

\node (S4) at (rpctrace Right Port -16) [send, anchor=east] {S};
\node (R4) at (rpctrace Left Port -16) [recv, anchor=west] {R};

\draw [port link] (Unknown Task sendright.east) -| ($(Unknown Task sendright.east)!.5!(R4.west)$) |- (R4.west);
\draw [internal link] (R4) -- (S4);

\draw [port link] (S4) -- (Task 4 recvright);

\draw [internal link] (S1) -- (R2);
\draw [internal link] ($(S1)!.5!(R2)$) -- (R1);
\draw [internal link] ($(S1)!.5!(R2)$) -- (R3);

\node (rpctrace message) [message] at ($(R4)!.5!(S4)$) {};

\node (rpctrace message recvright) [recv] at (rpctrace message) {R};

\endscope
\end{tikzpicture}
\end{frame}

\begin{comment}

% illustration of rpctrace problems with mach_port_insert_right()

\begin{frame}[fragile]
\begin{block}{}
\begin{semiverbatim}
\tiny
ports\_create\_port (port\_class, port\_bucket, sizeof (struct port\_info), \&readport);
readportname = ports\_get\_right (readport);
mach\_port\_insert\_right (mach\_task\_self (), readportname, readportname, MACH\_MSG\_TYPE\_MAKE\_SEND);
\end{semiverbatim}
\end{block}

\begin{block}{}
\begin{semiverbatim}
\tiny
routine mach\_port\_insert\_right(
                task            : ipc\_space\_t;
                name            : mach\_port\_name\_t;
                poly            : mach\_port\_poly\_t);

\end{semiverbatim}
\end{block}

\vfill

\begin{center}
\begin{tikzpicture}

\node (center point) {};

\task [left = of center point] {rpctrace}{}{lr}{
  \only<1-3>{
    \node (rpctrace Message) [message, anchor=north, opacity=0] {\scriptsize insert\_right};
  }
  \only<4-5>{
    \node (rpctrace Message) [message, anchor=north] {\scriptsize insert\_right};
  }
  \only<6>{
    \node (rpctrace Message) [message, anchor=north, fill=none, dashed, text opacity=0] {\scriptsize insert\_right};
  }
}

\only<4-5>{
  \node (rpctrace sendright) at (rpctrace Message) [send, anchor=west] {S};
  \node at (rpctrace sendright.west) [anchor=east, outer sep=0, inner sep=0] {(10,};
  \node at (rpctrace sendright.east) [anchor=west, outer sep=0, inner sep=0] {)};
}

\task [right = of center point] {Traced Task}{}{lk}{
  \only<1-2,5->{
    \node (Traced Task Message) [message, anchor=north, opacity=0] {\scriptsize insert\_right};
  }
  \only<3>{
    \node (Traced Task Message) [message, anchor=north] {\scriptsize insert\_right};
  }
  \only<4>{
    \node (Traced Task Message) [message, anchor=north, fill=none, dashed, text opacity=0] {\scriptsize insert\_right};
  }
}

\only<3>{
  \node (Traced Task sendright) at (Traced Task Message) [send, anchor=west] {S};
  \node at (Traced Task sendright.west) [anchor=east, outer sep=0, inner sep=0] {(10,};
  \node at (Traced Task sendright.east) [anchor=west, outer sep=0, inner sep=0] {)};
}

\draw [port link] (rpctrace Port 7) -| ($(rpctrace Port 7)!.5!(Traced Task Task Port)$) |- (Traced Task Task Port);

\node (S1) at (rpctrace Port 7) [send, anchor=east] {S};
\node (R1) at (rpctrace Port 4) [recv, anchor=east] {R};

\draw [internal link] (S1) -- +(-.5,0) |- (R1);

\draw [port link] (Traced Task Port 4) -- (rpctrace Port 4);
\node (S2) at (Traced Task Port 4) [send, anchor=west] {S};
\node at (S2.east) [anchor=west] {\scriptsize mach\_task\_self()};

\only<4>{
  \draw [message path] (Traced Task Message.north) |- (S2);
  \draw [message path] (Traced Task Port 4) -- (rpctrace Port 4);
  \draw [message path] (R1) -| (rpctrace Message.north);
}

\only<6>{
  \draw [message path] (rpctrace Message.north) |- (S1);
  \draw [message path] (rpctrace Port 7) -| ($(rpctrace Port 7)!.5!(Traced Task Task Port)$) |- (Traced Task Task Port);
  \draw [message path] (Traced Task Task Port) -| ($(Traced Task Task Port)!.5!(Kernel Message.west)$) |- (Kernel Message.west);
}

\only<2->{
  \node (R2) at (Traced Task Port -4) [recv, anchor=west] {R};
  \node at (R2.north) [anchor=south west, inner sep=2] {10};
}

\only<3>{
  \draw [right link] (Traced Task sendright) -- (R2);
}

\only<4>{
  \draw [right link] (rpctrace sendright) -- (R2);
}

\only<5->{
  \node (S3) at (rpctrace Port -4) [send, anchor=east] {S};
  \node (R3) at (rpctrace Port -1) [recv, anchor=east] {R};
  \draw [internal link] (S3) -- +(-.5,0) |- (R3);
  \draw [port link] (rpctrace Port -4) -- (Traced Task Port -4);
}

\only<5>{
  \draw [right link] (rpctrace sendright) -- (R3);
}

\only<1-5>{
  \node (Kernel Message) [message, anchor=south west, opacity=0] at (Traced Task.north) {\scriptsize insert\_right};
}
\only<6->{
  \node (Kernel Message) [message, anchor=south west] at (Traced Task.north) {\scriptsize insert\_right};
  \node (Kernel sendright) at (Kernel Message) [send, anchor=west] {S};
  \node at (Kernel sendright.west) [anchor=east, outer sep=0, inner sep=0] {(10,};
  \node at (Kernel sendright.east) [anchor=west, outer sep=0, inner sep=0] {)};
  \draw [right link] (Kernel sendright) -- (R3);
}

\end{tikzpicture}
\end{center}

\end{frame}

%% illustration of attach operation (two frames)

\begin{frame}
\begin{tikzpicture}

\node (center point) {};

\task [left = of center point] {rpctrace}{}{lr}{}
\task [right = of center point] {PUT}{}{l}{}

\queue [above = of rpctrace] {queue}{}

\node (below rpctrace) [below = of rpctrace] {};
\node (below PUT) [below = of PUT] {};

% invisible at first, but present to keep the canvas sizing consistent across all slides
\only<1-4>{
  \queue [white] {queue2}{at ($(below rpctrace)!.5!(below PUT)$)}
}
\only<5->{
  \queue {queue2}{at ($(below rpctrace)!.5!(below PUT)$)}
}

% invisible receive right to match one from the next set of slides, so the canvas sizing is identical
\draw [white, thick] (queue.west) -- ++(-3,1) node [recv, white, anchor=south] {R};

\only<1,5->{\node (R1) at (PUT Port 0) [recv, anchor=west] {R};}
\only<2-4>{\node (R1) at (PUT Port 0) [empty] {};}

\only<1-3,6->{
  \node (S1) at (R1.east) [send, anchor=west] {S};
  \node (S2) at (S1.east) [send, anchor=west] {S};
}

\only<1-3>{
  \draw [violet, thick] (PUT Port 0) -- +(-1,0) |- (queue.east);
}

\draw [violet, thick] (queue.north) -- ++(-1,1) node [send, anchor=south] {S};
\draw [violet, thick] (queue.north) -- ++(1,1) node [send, anchor=south] {S};

\only<2->{
  \node (R2) at (rpctrace Left Port 0) [recv, anchor=west] {R};
  \draw [violet, thick] (queue.west) -- ++(-2,0) |- (R2);
}

\only<3-8>{
  \node (S3) at (R2.east) [send, anchor=west] {S};
}
\only<9->{
  \node (S3) at (R2.east) [empty] {};
}

\only<7->{
  \node (S4) at (rpctrace Right Port 0) [send, anchor=east] {S};
  \draw [violet, thick] (rpctrace Right Port 0) -| ($(rpctrace Right Port 0)!.5!(queue2.west)$) |- (queue2.west);
}

\only<5->{
  \draw [violet, thick] (queue2.east) -| ($(queue2.east)!.5!(PUT Port 0)$) |- (PUT Port 0);
}

\only<8->{
  % S3's location moves between slides, so starting this line at S4 keeps the dashes from morphing between slides
  \draw [violet, dashed, thick] (S4) -- (S3);
}


\end{tikzpicture}
\end{frame}

\begin{frame}
\begin{tikzpicture}

\node (center point) {};

\task [left = of center point] {rpctrace}{}{lr}{}
\task [right = of center point] {PUT}{}{l}{}

\queue [above = of rpctrace] {queue}{}

\node (below rpctrace) [below = of rpctrace] {};
\node (below PUT) [below = of PUT] {};

% invisible at first, but present to keep the canvas sizing consistent across all slides
\only<1-3>{
  \queue [white] {queue2}{at ($(below rpctrace)!.5!(below PUT)$)}
}
\only<4->{
  \queue {queue2}{at ($(below rpctrace)!.5!(below PUT)$)}
}

\only<1-2,5->{
  \node (S1) at (PUT Port 0) [send, anchor=west] {S};
  \node (S2) at (S1.east) [send, anchor=west] {S};
}

\only<1-2>{
  \draw [violet, thick] (PUT Port 0) -- +(-1,0) |- (queue.east);
}

\draw [violet, thick] (queue.north) -- ++(-1,1) node [send, anchor=south] {S};
\draw [violet, thick] (queue.north) -- ++(1,1) node [send, anchor=south] {S};
\draw [violet, thick] (queue.west) -- ++(-3,1) node [recv, anchor=south] {R};

\only<2->{
  \node (S3) at (rpctrace Left Port 0) [send, anchor=west] {S};
  \draw [violet, thick] (queue.west) -- ++(-2,0) |- (S3);
}

\only<4->{
  \node (R3) at (rpctrace Right Port 0) [recv, anchor=east] {R};
}

\only<4->{
  \draw [violet, thick] (rpctrace Right Port 0) -| ($(rpctrace Right Port 0)!.5!(queue2.west)$) |- (queue2.west);
}

\only<5->{
  \draw [violet, thick] (queue2.east) -| ($(queue2.east)!.5!(PUT Port 0)$) |- (PUT Port 0);
}

\only<6->{
  \draw [violet, dashed, thick] (S3) -- (R3);
}


\end{tikzpicture}
\end{frame}

\end{comment}

\end{document}
