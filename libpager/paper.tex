
To support a multi-node Hurd cluster, the libpager used by filesystem
translators needs to support multiple clients (i.e, multiple kernels
mmap'ing files).  Hurd's original libpager, written in the mid-1990s,
only supported a single client, and was thus suitable for single node
operation.

This is a crucial library.  It provides the cache coherence logic
necessary to implement inter-node shared memory, whether disk-based or
not.  Its use in the tmpfs translator, which is in turn used by
Hurd's POSIX shared memory implementation


CAP THEOREM
  states that we can't have all three: consistency, availability, partitioning
  right now, we sacrifice availability - if the network partitions, libpager can deadlock
  if we introduce a timeout on lock requests, then we're sacrificing consistency


GENERAL NOTES

pager_read_page() is expected to allocate memory (unless it returns an
error) and pass it to libpager, which is responsible for deallocating
it (actually by passing it to the kernel).

pager_write_page() does no memory allocation or deallocation.  It
receives allocated memory and leaves it allocated, with libpager
responsible for dellocation (or passing it to the kernel).

pager_flush() and pager_flush_some() discard dirty pages!

pager_return() and pager_return_some() flush and return dirty (and precious) pages.

pager_sync() and pager_sync_some() return dirty (and precious) pages but don't flush.

The current kernel implementation seems to request pages one at
a time, as the process attempts to access them.

If we're servicing a page read-only to multiple clients, where do we
get the data from?  If our local kernel has a copy, we almost
certainly want to get it there, since it's just a manner of flipping
bits in a hardware page table to obtain a new "copy" of the data.
Otherwise, we probably want to get the data from the local disk, but
if the disk is very busy and we've got fast network connections, we
might want to get that data from another node in the cluster.

Mach's current memory object API makes this a lot more difficult.  You
can't just request that the kernel hand you back an unmodified page,
except by flagging it "precious".  That's not too bad if the kernel
and the memory manager are both local on one node, but if you do this
over the network, you're sending unnecessary copies of the memory
pages around.

How do we figure out which client is the local kernel?  One possible
solution is for netmsg to listen on /servers/netmsg for an RPC that
accepts a send right and identifies which node currently hosts the
corresponding receive right.

[[ the kernel uses different control ports for different memory objects ]]

  If the kernel uses the same control port for all of its pager
  operations, then we wouldn't have to make this query for every memory
  object; a single ext2fs translator, for example, could globally
  identify the location of all of the control ports, one for each node,
  and this information could be shared amongst all of that translator's
  pager objects.

DISTRIBUTED LIBPAGER.  We could use a shim process on remote nodes, to
allow requests for an existing page go to a node that holds a copy of
that page, instead of to the node with the disk.  Two ways I can
imagine this.  Either ext2fs forks off multiple processes on different
nodes, or a mechanism is developed to detect when a port is remote,
figure out which node it (currently) resides on, and then libpager can
fork a process and push it to the other node, invisible to ext2fs
proper.


CONFIGURATION OPTIONS

- do we write a page back every time we handle it?
  perhaps always call write_page, and add a flag to write_page indicating that we still have a copy?
- how long do we let clients hold a page if other clients are waiting for it?
  currently: 0 (immediately send lock requests)
  could create deadlock if the kernel returns pages without making any forward progress
- how do we obtain pages when we can read them from multiple sources?
  only applies to READ access; local kernel should be preferred
- handle network partitions / failed nodes / malfunctioning kernels
  timeout while waiting for a client?
  CAP theorem - timeouts would sacrifice consistency to achieve availability


POSIX SHARED MEMORY

Hurd's System V shared memory implementation, in glibc's
sysdeps/mach/hurd, operates by mapping "files" in the /run/shm portion
of the filesystem tree.  /run/shm is typically implemented using the
tmpfs translator, which is linked with libpager.  Thus, a multi-client
libpager enables the use of inter-node shared memory.  What is
required to make this happen is a shared /run/shm - shared with full
Mach IPC semantics, not just NFS - so netmsg is required.  unionfs can
possibly be used to overlay the various /run/shm's on each other.


ON REMOVING pager_offer_page()

Brent Baccala wrote:

> I've reviewed the existing code, and I have a problem with the
> function pager_offer_page().

> First, the API is problematic, since pager_offer_page() is a call from
> the memory server (i.e, ext2fs) into the libpager library, instructing
> the library to offer a page to the client (i.e, the kernel) that
> hasn't been solicited by the client.  The problem is that the function
> parameters don't indicate which client to offer the page to.

> Second, I can't find anywhere in the hurd source tree where this
> function is actually used.

> Third, why would we want this feature?  Why would a memory server ever
> want to send an unsolicited page to a client?

> So, I propose deprecating pager_offer_pager() by simply removing it
> from the library.

> Any objections?

Kalle Olavi Niemitalo <kon@iki.fi> wrote:

> Commit 84cf9c0f312637b670cc87224ff7e7c4da659e36 on 2013-09-17
> removed the ufs directory, in which the offer_data function used
> to call pager_offer_page.  The last argument of pager_offer_page
> always pointed to a page that was part of the global zeroblock,
> which was the size of a filesystem block and filled with zeroes.

> One of the offer_data calls was preceded by a comment:

>   /* Make sure that any pages of this block which just became allocated
>      don't get paged in from disk. */

> I don't know how ext2fs makes sure of that, or whether
> pager_offer_page might again be needed for the same purpose in
> the future.


C++ PAGEMAP

What's not so clear is how to code the pagemap itself.  A straight-up
array of pointers?  Easy enough to understand, but then we have to
code reference counting to know when those dynamic structures can be
deallocated.  An array of shared_ptr's?  Simplifies the code, because
it does all the reference counting for us, but at the cost of making
the array four times bigger than it needs to be.  A custom class with
a clever copy constructor?  Then we just assign into the array and the
copy constructor takes care of either finding an existing pagemap
structure or creating a new one, at the cost of obscuring the fact
that "pagemap[i] = new_client_list;" is far more complex than a simple
assignment.

Right now, I use the clever copy constructor.  All the queues and
lists are private, and we've got a bunch of methods to access them,
allowing us to change the pagemap structure later if we want.

What I want to avoid is to start looking at that code, saying "coping
all these queues and lists is inefficient" (it is), and start adding
all kinds of C++ tricks to speed it up.  Just write it so that it's
simple and clear, and if performance becomes an issue, revisit it
later.


Richard Braun <rbraun@sceen.net>:

> From my fifteen years or so experience working in various projects with
> C++ and other languages, my opinion is that C++ is never the best
> choice. It makes you trade lines of code for increased complexity in
> understanding and following the code. You should either go with C for
> excellent control, or another higher level langguage for e.g. user
> interfaces and tasks that don't require the highest performance.

> I really don't think the problem you describe would be so hard to solve
> in C.


NEW PAGETABLE

struct pagemap_entry {
   /* ACCESSLIST */
   port_t clients_with_access[];
   bool write_access_granted;

   /* WAITLIST */
   /* A list of the clients waiting for access, in order of arrival, and what kind of access they are waiting for */
   port_t clients_waiting[];
   boolean clients_waiting_for_write[];

   /* NEXTERROR list */
   /* A list of outstanding error codes, paired with clients, returned during unlock requests. */
   /* They'll be transmitted to the client that requested them. */

   /* WRITEWAIT list */
   /* A list of pages waiting to be written out to disk */

   ERROR code - a kern_return_t, so we're not limited to a small number of errors

   /* Flags */
   PAGINGOUT - set this flag when we get a data return, clear it after the write is finished
   INVALID - set if the data on disk is invalid because we got an error writing the page back to the filesystem
      If some clients still have copies of the data, we may be able to operate for a while, then retry the write operation
}


PAGEMAP FOR AN ARBITRARY NUMBER OF CLIENTS

How to maintain the pagemap for a large number of clients?  Each page
has a pointer to a structure indicating which clients currently hold
copies of the page.  We anticipate that many pages (say, all the pages
in a single shared library) will have the same set of clients, so
there's one structure for each combination of clients, with multiple
pages pointing to it.  Pagemaps can be potentially large, so a pagemap
entry should be a single pointer.

Each file corresponds to a separate memory object, and separate memory
objects get separate control ports in the kernel.  A future
enhancement could be for the kernel to use a single control port for
all of its memory objects, allowing libpager to reduce the number of
these combination structures, if different files have the same usage
pattern (a program and some of its shared libraries, for example).

Each pagemap entry has an ACCESSLIST and a WAITLIST.  ACCESSLIST is
unordered; WAITLIST is ordered.  ACCESSLIST is stored sorted.

What happens when either list gets a new client (or loses one)?  We
have to search in the structures for the new combination, and create a
new structure if the desired combination doesn't exist yet.

We currently use a C++ std::set and let it do the searching for us.

Other possibilities:

  How can we organize the structures?  First, by the number of clients
  on each list.  We'll always know how many clients are in the
  ACCESSLIST we're searching for, so if we've got six ACCESSLIST clients
  and are moving the only client on WAITLIST to ACCESSLIST, we'll search
  in the seven-client ACCESSLIST, zero-client WAITLIST tables.  We can
  use a hash table or a tree.

  Because ACCESSLIST is sorted, we can binary search to find an
  ACCESSLIST that matches the one we want, then linear search forward
  and backward to find the one we want.

  or... Keep ACCESSLIST and WAITLIST completely separate for each other.
  Each pagemap entry has an ACCESSLIST pointer and a WAITLIST pointer,
  as well as some flags, including WRITE-ACCESS, which indicates if
  we've given out write access to a single client, in which case
  ACCESSLIST isn't a pointer, but a single client port number.  If
  WRITE-ACCESS is false, then ACCESSLIST points to a NULL-terminated
  list of client port numbers.  The disadvantage here is that the
  pagemap entries require more space (at least two pointers).

  WAITLISTs are ordered and include a flag for each client indicating
  whether it's waiting for READ access or WRITE access.  A special case,
  indicated by a flag, could be a single client on a WAITLIST, in which
  case we don't need a pointer to a WAITLIST, but rather can use the
  port number instead of the pointer.

The pagemap structures do not count the number of pages pointing to
them.  If we did that, it would facilitate deallocation when the count
drops to zero, but would interfere with the "add table" pointer
optimization described below.


POINTER OPTIMIZATION (UNIMPLEMENTED)

To speed this process, each structure can maintain pointers to
accelerate this mapping.  Each structure would have an "add table"
would maintain mappings like "add client 123 -> structure 0x80fadde0".
If the "add table" doesn't have an entry, then we need to search all
the structures to find one.  A problem with this is the need to
invalidate those pointers when a structure is deallocated.  Another
problem is that it requires the structures to resize when these
pointers are added. Keep this idea for a future enhancement.

single client pointer optimizations
  1. no access ->2 ->3
    pagemap_entry * put_client_1_on_waitlist_for_READ
    pagemap_entry * put_client_1_on_waitlist_for_WRITE
  2. waiting for read ->4
    pagemap_entry * move_first_WAITLIST_client_to_ACCESSLIST
  3. waiting for write ->5
    ditto (move_first_WAITLIST_client_to_ACCESSLIST)
  4. has read ->6
    ditto (put_client_1_on_WAITLIST_for_WRITE)
  5. has write
  6. has read; waiting for write ->5
    ditto (move_first_WAITLIST_client_to_ACCESSLIST)


PAGEMAP FOR A SMALL NUMBER OF CLIENTS (UNIMPLEMENTED)

First, steal a bit from the pointer by aligning the structure on a
word boundary.  Then, if the least significant bit is 1, interpret the
pagemap entry as a bitfield rather than as a pointer to a more
complicated structure.

A 32 bit field (err... make that a 31 bit field) can handle up to four
clients.  The new pagemap entries have an ACCESSLIST, which is an
unsorted set of clients (those who currently have access), and a
WAITLIST, which is a sorted list of clients that have requested
access, and a few boolean flags.

We number our clients 0 through 3, and I'm thinking above moving their
ports into high port number space to easily identify them.  So, the
first client on the first memory object would be on port 0x8000000,
the second client would be on 0x80000001, etc.  The second memory
object would have its ports on 0x80000004 through 0x80000007.  So we
can just strip off the lowest 2 bits to figure out our "client
number".  I don't think protected payloads would help, since we're not
receiving messages on these ports; they're send ports used to
communicate with the kernel(s).

ACCESSLIST entries requires two bits - one to indicate if access is
granted, and one to indicate if it's read-only or read-write access.
So, we use 2*4 = 8 bits for the ACCESSLIST.  The WAITLIST is sorted
(client requests are processed FIFO) and requires four bits per slot -
one to indicate that the slot is in use, two to identify the client,
one to indicate if the request is for read or write access.  We need
four slots for four clients, so 4*4 = 16 bits for the WAITLIST.  Then
we need a few more flag bits.  It all fits.

Once we hit our fifth client, or some oddball condition (like an error
return other than the expected EIO, ENOSPC, EDQUOT), we shift to a
pointer that points to a more complicated, dynamically allocated
structure.


ACCESS CONTENTION

Putting a client on an empty queue triggers lock requests to all
clients in the working set, with the exception of the requesting
client itself, if it's already got read access and is requesting write
access.  Once the working set size drops to zero (or one), then the
original request can be satisfied.

In the meantime, additional clients can be added to the queue.  Each
client should be allowed access to a page for a period of time before
a lock request is sent.  Is the easiest way to do this to use the
timeout parameter on mach_msg?

Current implementation waits until the kernel returns a page, then
writes it to disk.  If a request for the page comes in before the
write completes, we wait for the write to finish, then send it right
back out without needing a read.

The new implementation will be handling pages as they're passed from
one client to another.  Should they be written out each time they are
handled?  Not write them at all until they're finally released?  Have
some kind of timer or counter to write them at a controlled pace?
Initially, we'll not write them at all until they're finally released,
which mimics the current behavior.

So when the last client returns a page, we look to see if anything is
in the wait queue.  If so, the first client(s) in the queue get the
pages.  If not, we flag that a write is in progress and start it.
When it completes, we check the queue again to see if anything is now
waiting.  If so, we service it.  If not, we discard the data and
notify the user that the page has been evicted.



PSEUDOCODE

service_waitlist: (pass in a pagetable pointer, a data pointer, a read/write flag, a deallocate flage)
  XXX: data length must be page_size, as this logic is for a single page
  call with pager locked
  use m_o_data_supply to supply first set of WAITLIST clients
    all but last specify deallocate=false; last one specifies deallocation flag as passed in
  remove those clients from WAITLIST and move them to ACCESSLIST
  what if WAITLIST still has clients on it after sending messages?
     [ ] use a timeout before sending more lock requests?
     [X] send lock requests right away to everything on ACCESSLIST

send_error_to_WAITLIST (pass in an error and offset):
  XXX: data length must be page_size
    ( XXX this causes a data_error to be sent for each page, when we could consolidate them, )
    ( but hopefully data_error is a corner case.  For ENOSPC or EDQUOT, though, not really. )
  send m_o_data_error's to everything on WAITLIST and clear WAITLIST
    anything on WAITLIST that's also on ACCESSLIST should get a lock instead of data_error
      and be put on NEXTERROR list
  (such a client sent an unlock request and is waiting for a lock, not data_error)

  [ ] flush everything else on ACCESSLIST
  [X] do nothing to ACCESSLIST
  [ ] do something more clever with stuff on ACCESSLIST
     (we got a read error, but other clients have copies of the data)

finalize_unlock: (pass in a page number and an error code)
  if page not flagged ERROR:
     send m_o_lock_request to first client on WAITLIST (no reply), remove this client from WAITLIST,
     and upgrade its ACCESSLIST entry to reflect WRITE access
     if there are additional clients on WAITLIST:
        [ ] use a timeout before sending a lock request to this client
        [X] send a lock request right away to this client
            (XXX note that we just sent a lock request to answer the unlock; really want a timeout here)
  else (page flagged ERROR):
     send m_o_lock_request to first client on WAITLIST (internal flush variant)
     remove this client from WAITLIST
     add this client and error to NEXTERROR list
     (wait for reply and don't remove from ACCESSLIST until we get it)
     (other clients on WAITLIST will be processed when the lock is answered)

m_o_data_request: (kernel requesting a single page)

  if this client is on NEXTERROR list and WRITE ACCESS WAS REQUESTED, send a m_o_data_error message,
        move NEXTERROR to ERROR, remove the NEXTERROR list entry, and return
  if the page is PAGINGOUT:
    if WAITLIST is empty and ACCESSLIST is not, send lock request (internal flush variant) to ACCESSLIST client
       ASSERT: there should only be one WRITE client on ACCESSLIST - the client that returned the data we're paging out
    add requesting client to WAITLIST and return
  if this client is already on ACCESSLIST, it flushed and is trying to re-aquire access,
     so check to see if WAITLIST is empty
     if so, remove client from ACCESSLIST and proceed (kernel flushed without being asked)
     if not, we're flushing, so add client to WAITLIST, run internal_lock_completed on this page, and return
  if WAITLIST is not empty, add requesting client to WAITLIST and return
  if another client has WRITE access (WAITLIST is empty), add requesting client to WAITLIST,
     send lock request (internal flush variant) and return
  if WRITE access is requested and ACCESSLIST is not empty (other clients have access),
     add requesting client to WAITLIST (it's empty),
     send lock requests (internal flush variant) and return
     (we could give out the page with READ access and wait for a unlock request, but I think not)
  else
     (ACCESSLIST is empty or READ access is requested and only READ clients are on ACCESSLIST)
     if page is flagged INVALID, send m_o_data_error to this client and return
     add this client to WAITLIST (it's empty)
     unlock pager
     read the page with pager_read_page()
     relock pager
     if pager_read_page() returned an error, mark page with ERROR and call send_error_to_WAITLIST
     if no error, service_waitlist(deallocate=true) (read/write status supplied by filesystem), clear ERROR

m_o_lock_completed:
  m_o_lock_completed could have been sent in response to an internal lock request, or a fs flush, sync, return
  find client/offset/length in list of lock requests and decrement locks_pending
  if locks_pending is zero:
    if internal_lock_outstanding, and WAITLIST is not empty, then call internal_lock_completed
    remove client/offset/length from list
  if no more clients for this offset/length, wake up any waiting thread

internal_lock_completed: (internal flush variant)
  use alloca() to allocate an array of flags, one set (in a uint8) for each page in current message:
    PAGEIN, UNLOCK, ERROR
  for all pages covered by message:
    [ if this client had WRITE access, print error (should have used m_o_data_return instead of m_o_lock_completed) ]
    if this client isn't on ACCESSLIST, do nothing (m_o_data_return already processed the data)
    if WAITLIST is empty, do nothing (m_o_data_request already called internal_lock_completed)
    else:
      remove client from ACCESSLIST
      if ACCESSLIST is empty but WAITLIST is not:
        if INVALID is not set: set PAGEIN flag
        else (INVALID set): call send_error_to_WAITLIST
          XXX: assumes that ERROR is correctly set whenever INVALID is set

      if ACCESSLIST has a single client with READ access and it's also the first client on WAITLIST requesting WRITE access,
        set UNLOCK flag
      otherwise, if ACCESSLIST has multiple clients, or a single client that's not first on WAITLIST,
        then we're still waiting for them to answer their locks, so do nothing
  unlock pager
  use alloca() to allocate an array of pointers and write lock flags, one of each for each page in current message
  for each page flagged PAGEIN:
    read the page with pager_read_page(), saving resulting pointer and write lock flag
    if pager_read_page() returned an error, set flag ERROR
    (a client with no access requested WRITE access and had to wait for other clients to flush)
  for each page flagged UNLOCK:
    call pager_unlock_page(), and set ERROR based on return value
    (a client with READ access requested WRITE access and had to wait for other clients with READ access to flush)
  relock pager
  for each page flagged PAGEIN:
    if page not flagged ERROR:
       service_waitlist(deallocate=true) (read/write status supplied by filesystem), clear ERROR
       XXX: this causes multi-page operations to broken up into single page ops
    else (page flagged ERROR): call send_error_to_WAITLIST
  for each page flagged UNLOCK: call finalize_unlock

m_o_data_unlock: (kernel requesting write access when it's already got read access)
  lock pager
  alloca() an array of UNLOCK and ERROR flags, one pair for each page in message
  for all pages:
    ASSERT: this client already on ACCESSLIST with read access
    ASSERT: PAGINGOUT flag is not set (only should be set if a client had WRITE access)
    if WAITLIST contains at least one client waiting for WRITE access, do nothing and return
      (we're either already trying to flush everything on ACCESSLIST, or will, including this client)
      (if it saw the flush request before sending the unlock, it would never have sent the unlock)
      (so it hasn't processed the flush yet, and will interpret it as a response to the unlock)
    else if WAITLIST contains only clients waiting for READ access, add client to WAITLIST for WRITE and return
      (ACCESSLIST contains only READ clients and WAITLIST contains only READ clients)
      (so we're waiting for a page in to finish, as it will trigger a flush that will answer this unlock)
    otherwise, if there's other clients on ACCESSLIST, add client to WAITLIST for WRITE (it's empty)
      and send lock requests to other clients (internal flush variant) and return
    otherwise, we're the only client on ACCESSLIST and WAITLIST is empty:
      add client to WAITLIST for WRITE
      set UNLOCK flag
  unlock pager
  for all pages with UNLOCK flag set:
    call pager_unlock_page() and set ERROR based on return value
  relock pager
  for all pages with UNLOCK flag set: call finalize_unlock

lock_object: (several types: flush, sync, return, response to an unlock request)
  NEW MODE NEEDED: asynchronous, but reply requested
  wrapper around m_o_data_lock message
  if sync requested, add to lock_requests list and increment count
     after m_o_lock_request message sent, wait for locks_pending and pending_writes to drop to zero,
     then decrement threads_waiting and remove lock_requests if zero

  lock_object() called by pager_ calls to flush, sync, and return data from kernel,
    as well as m_o_data_unlock

  if a m_o_data_return comes in while waiting for a synchronous lock to complete,
    we wait for the write to finish before returning from the lock
    the data return could be triggered by the lock request (i.e, a sync or a return)
  if a lock request comes in while a write is happening, we don't do anything to synchronize them

m_o_data_return: (kernel returning pages)
  (this can not be used in liu of a lock completed message for a WRITE,
     because it doesn't indicate if the kernel still maintains WRITE access)
  npages = length / __vm_page_size;
  pm_entries points to 'npages' entries in pagemap
  lock pager
  ASSERT: this client should be all the page's ACCESSLISTs

  if kernel_copy is false:

    remove this client from all of these page's ACCESSLISTs
    [ ASSERT: these ACCESSLISTs should now be empty ]
    [ ASSERT: this client is not on any of the WAITLISTs (it had WRITE access, so there's nothing for it to wait for) ]
    [ not necessarily - if the pages are precious, they might be coming back with only READ access ]
    alloca() an array of UNLOCK, NOTIFY, and ERROR flags, one set for each page in message
    for all pages:
      if ACCESSLIST is empty and WAITLIST is not empty:
        service_waitlist(deallocate=false)
        (a client with no access requested WRITE access and the last client we were waiting for flushed)
      if ACCESSLIST has a single client with READ access and it's also the first client on WAITLIST requesting WRITE access,
        set UNLOCK flag
        (a client with READ access requested WRITE access and had to wait for other clients with READ access to flush)
      if both ACCESSLIST and WAITLIST are empty and pages are not DIRTY:
        set NOTIFY flag
        ( DIRTY pages will get their notifications done after their pager_write_page() calls )
    if any pages required UNLOCK or NOTIFY:
      ( this can't happen if pages are dirty )
      ASSERT: ! pages_dirty
      XXX: should this code be moved down further?
      unlock pager
      for all pages with UNLOCK set:
        call pager_unlock_page() and set ERROR based on return value
      for all pages with NOTIFY set:
        call pager_notify_evict()
        clear INVALID, ERROR, and NEXTERROR for this page
      relock pager
      for all pages with UNLOCK set: call finalize_unlock

  if pages are dirty:

    ASSERT: if kernel_copy = true, then this client is the only thing on ACCESSLIST
    ASSERT: if kernel_copy = false, then ACCESSLIST is empty
      (this client was the only thing on ACCESSLIST, because it had WRITE access to create a dirty page)

    [ ] for all pages with empty WAITLISTs and ACCESSLISTs:
        (XXX pager_sync needs to trigger a write even for pages with active clients)
    [X] for all pages in message:
         set PAGINGOUT in pagemap

    [ ] if PAGINGOUT was already set for any page, add this message, including kernel_copy flag
        and the pointer to the last lock incremented, to WRITEWAIT list and return
        ( allow simultaneous pager_write_page() calls for different pages )
    [X] if WRITEWAIT list is not empty, add this message (and kernel_copy and pointer) to WRITEWAIT list and return
        if WRITEWAIT list is empty, add this message (and kernel_copy and pointer) to WRITEWAIT list
           and process everything on WRITEWAIT list
        ( allow no simultaneous pager_write_page() calls )

  else (pages are not dirty):
    munmap() data

service_first_WRITEWAIT_entry:
  use alloca() to allocated an array of flags, one set (in a uint8) for each page in current message:
    NOTIFY, ERROR, and PAGEOUT
    ( perhaps put this is a subroutine so that this array is deallocated after every message is processed )
  for all pages in current message:
    search WRITEWAIT list for a matching page
    if no match, set PAGEOUT flag
    ( this avoids writing a page if we've got more recent data waiting to write )
    ( this could be triggered by a rapid succession of pager_sync() calls on a busy page and a slow disk )
  unlock the pager
  call pager_write_page() on all pages flagged PAGEOUT
    set ERROR on any error return
  relock the pager
  for each page we just wrote:
    search WRITEWAIT list for matching pages
    ( can't use PAGEOUT flag because WRITEWAIT list might have changed while pager was unlocked )
    set INVALID equal to ERROR
    if something found on WRITEWAIT:
      ASSERT: kernel_copy is true (else how could there be a later write?)
      do nothing
    else if none found and WAITLIST is not empty:
      clear PAGINGOUT
      (this client had WRITE access because the page was dirty; so everything on WAITLIST is a data request)
      if kernel_copy is false: service_waitlist (deallocate = false)
      (if kernel_copy is true then we're still waiting for a flush to complete)
    else (none found and WAITLIST is empty):
      clear PAGINGOUT
      if kernel_copy is false, set NOTIFY flag and clear ERROR and NEXTERROR

  munmap() current message

  unlock the pager
  if client requested notify_on_evict, call pager_notify_evict() for any page flaged NOTIFY
    also, clear INVALID, ERROR, and NEXTERROR for these page
    ( the kernel didn't keep a copy and it didn't get shipped back out in an m_o_data_supply message )
  relock the pager (this can be cleverly wrapped into previous unlock/lock cycle)

  signal wakeup on anything waiting on this WRITEWAIT entry
  remove this WRITEWAIT entry from WRITEWAIT list

pager_sync:
  only needs to do anything if a client has WRITE access
  might need to do multiple locks if different clients have WRITE access to different pages

pager_return:
  scan pagelist for specified range and form the union of all clients on the ACCESSLISTs
  send multiple locks for multiple clients
  set (or increment) locks_pending by the number of lock requests sent

pager_flush:
  scan pagelist for specified range and form the union of all clients on the ACCESSLISTs
  send multiple locks for multiple clients
  set (or increment) locks_pending by the number of lock requests sent

m_o_terminate:
  mach_port_destroy() control and name ports passed in
    (we had send rights and should have just received receive rights in the m_o_terminate message)

pager_shutdown:
  set terminating flag
  send pager_return (flush, return all, sync) to all active clients
  in the meantime:
    ignore all m_o_data_request and m_o_unlock requests
    m_o_lock_completed messages: don't do "internal" processing, but do notify threads
    process m_o_data_return messages, but don't generate any lock requests or data supply messages as a result
    [ ] handle new m_o_init_object messages by adding them to client list, but not sending m_o_ready
        that way, they'll get m_o_destroy in response
    [X] ignore m_o_init_object messages
  [ "De-allocating the abstract memory object port also has this effect" - Mach Kernel Principles p. 43 ]
  [ so... we don't have to do this section ]
  [ send m_o_destroy (error = ENODEV) to all active clients
  [   now handle new m_o_init_object messages by sending m_o_destroy in response and adding to client list
  [ wait for m_o_terminate replies from all clients
  call ports_destroy_right, which will blow away our receive right, and ultimately call the destructor,
    which will then call drop_client on any remaining clients

SHUTDOWN CODE

Old libpager would flush everything (no special flags set), then
destroy the receive right and ignore any final messages.  This created
a minor race condition where clients could request a page while the
flush was processing, which would then get dropped.


MISTAKES AND OMISSIONS IN CURRENT INFO DOCUMENTATION:
  - pager_notify() undocumented
  - pager_read_page() / pager_write_page() / pager_unlock_page() don't discuss re-entrance of these routines
  - "deallocate" argument missing in m_o_data_supply


POTENTIAL PROBLEMS

BUG IN OLD CODE (NOT):
  client has WRITE access and page is being actively changed
  pager_write() is relatively slow
  pager_sync() is called asynchronously three times in rapid succession
  client returns data three times
  first m_o_return_data starts a pager_write and sets PAGINGOUT
  next two m_o_return_data's both see PAGINGOUT and set WRITEWAIT
  first pager_write finishes, sees WRITEWAIT, and wakes other two threads
  which one goes first?  coin flip
  thus, there's a chance that the second and third m_o_return_data's will be processed in reverse order
  actually, current implementation avoids this problem
    The second and third m_o_return_data's are both queued by the demuxer and don't
    get serviced until the first one finishes.  Then they are processed in order.

BUG IN OLD CODE:
  m_o_data_supply is called in data-return.c with precious = 0, instead of notify_on_evict

BUG IN OLD CODE:
  in pager_shutdown(), pthread_mutex_unlock() is called after ports_destroy_right(), which
  might have free'd the struct pager

CAVEAT IN OLD CODE:
  if we get a write error during a m_o_data_return, and we're flagged PAGEINWAIT because of an outstanding READ request,
  then the INVALID flag doesn't get set, but after we hand out the data, we won't get it back from the kernel
  (assuming that notify_on_evict is FALSE and thus pages are not flagged PRECIOUS)
  then we've dropped data without flagging anything; next read request will read old data from the disk

BUG IN OLD CODE (NOT):
  page 1 is dirty
  pager_sync() is called
  page 1 is returned and begins PAGINGOUT
  pages 1 and 2 are changed and become dirty
  pager_return() is called
  pages 1 and 2 are returned; m_o_data_return sees that page 1 is PAGINGOUT, flags it WRITEWAIT and sleeps
  page 2 is requested by the kernel
  page 2 has no flags set, so a pager_read_page() is started, even though more recent data is waiting to be written
  currently, serialization in demux prevents this problem, too, as the page 2 data_request won't start
    processing until the earlier data_return's complete


ISSUE:
  client on WAITLIST could have sent an unlock request and be waiting for a lock, not data_error
  a unlock request will come from a client on ACCESSLIST (for READ) and will have it put on WAITLIST (for WRITE)
  there could be an outstanding flush if something is already on WAITLIST for WRITE

RESOLVED ISSUE:
  data_request logic could cause multiple overlapping calls to pager_read_page() if a bunch of clients
    request READ access near simultaneously

RACE CONDITION:
  client requests access
  client is put on WAITLIST and pager_read_page() begins
  another client requests contradictory access via m_o_data_request
  NOT [ we send a lock request to original client, even though it hasn't gotten the data yet! ]
  NOT [ then pager_read_page() completes, and we send the data to the original client! ]
  YES [ no messages are sent since WAITLIST is not empty ]
  YES [ new client is put on the end of WAITLIST ]

RACE CONDITION:
  client B has READ access
  client A requests WRITE access (m_o_data_request)
  lock (flush) request is sent to client B and client A is added to WAITLIST
  nearly simultaneously, client B requests WRITE access (unlock)
  do nothing, since client B wouldn't have sent unlock request if it had already received lock request,
    so it will interpret the lock request as an answer to the unlock request

RACE CONDITION:
  client A has WRITE access
  pager_return()  (pager_sync would have kernel_copy = true; pager_flush never returns data)
  client A is sent a lock request
  client B requests WRITE access
  client A is sent another lock request, and client B goes on WAITLIST
  client A sends back page (kernel_copy = false)
  client B gets serviced
  client A sends completions for both lock requests
  RESULT: pager_write_page() is never called; data just moved from client A to client B
  ( old code would always call write_page(), then set PAGEINWAIT flag on request, then service request after page out finished)


Q: when could a client on WAITLIST be waiting for an lock, not a data_supply?
A: when that client already had READ access and sent an unlock request to get WRITE access
  if a client sent an unlock request, then it already had READ access
  how could this happen if there's (other) clients on WAITLIST?
  1. a lock/flush message has already been sent to service the clients on WAITLIST,
       the client sent the unlock request before processing the lock/flush,
       so essentially the unlock has already been answered and can be ignored
     this happens if a client is waiting for WRITE access when other clients had access (of some kind)
  2. the clients are on WAITLIST because we're waiting for paging (in or out) to complete
     in this case, we need to send a lock

RACE CONDITION:
  client A has READ access
  client B requested READ access, and we start paging in (client B goes on WAITLIST)
  client A requested WRITE access (via unlock), so it goes on WAITLIST
  client B's pager_read_page() returned an error, so we want to send data_error to B
     and lock to client A, with a queued NEXTERROR for client A


RACE CONDITION: (not sure about this - idea is to trigger something very similar to what is matched in above code)
  client A has WRITE access
  client B requests READ access
  lock request is sent to client A
  client A returns page with data_return, and client B gets the data and goes on ACCESSLIST
  client B returns the data and is removed from ACCESSLIST
  client C requests READ access, goes on WAITLIST and a pager_read_page() starts
  client A's lock completed message is processed, triggering a second pager_read_page(), since ACCESSLIST is empty

RESOLVED ISSUE: if multiple clients with READ access attempt to unlock near simultaneously,
  data_unlock logic will trigger multiple lock requests (a set for each client attempting to unlock)


ISSUES:
  before pager_write_page() returns, we can't call pager_read_page() or pager_write_page()
  after pager_write_page() returns, we might have added additional clients to WAITLIST,
    but did we add them while waiting for pager_write_page() or were they there before?
    if ACCESSLIST clients have READ access, and first WAITLIST clients are requested READ access, they are new
    if ACCESSLIST clients have READ access, and first WAITLIST client is requesting WRITE access,
      they may or may not be new
    if ACCESSLIST client has WRITE access, and WAITLIST isn't empty,
      they may or may not be new
    if we wait until pager_write_page() returns to send lock requests, we avoid having to figure out
      if we've already sent lock requests, and introduce some delay to let clients use the data

HOW ABOUT:
  before calling pager_write_page(), service the first batch of clients on WAITLIST
     and move them to ACCESSLIST, but send no lock requests
  after pager_write_page(), service any compatible WAITLIST clients and move them to ACCESSLIST
  if there are still outstanding WAITLIST clients, send lock requests to everyone on ACCESSLIST

RACE CONDITION:
  client A has WRITE access
  pager_return() is called; a lock request is sent to A
  client B requests access; a lock request is sent to A
  client A answers the pager_return's lock request with a data_return (kernel_copy = false; dirty = true)
  data_return code services WAITLIST and sends data to B
  second lock requests answers and does nothing since A is no longer on ACCESSLIST
  should be OK

RACE CONDITION:
  client A has WRITE access
  pager_sync() is called asynchronously
  client A syncs with a m_o_data_return, but keeps a copy of the data
  near simultaneously, client B requests access and libpager requests a flush
  client A answers with a new m_o_data_return containing data different from the first
  libpager sees the first m_o_data_return and hands it out to client B
  RESULT: client B does not have the most recent changes to the data
  [ ] SOLUTION 1: if m_o_data_return indicates kernal_copy=true, we can't hand it out to other clients
         PROBLEM: m_o_data_return doesn't indicate if the kernel has released WRITE access,
                  so this doesn't work if client B is requesting READ access and the lock
                  request is only to revoke WRITE access
  [X] SOLUTION 2: when we request a flush, wait for a lock_completed message


LOCK SEQUENCE:

client had READ access
FLUSH LOCK
NON-FLUSH LOCK (sync)
FLUSH LOCK COMPLETED (waits for non-flush lock before processing)
DATA REQUEST from flushed client for READ access - it's on ACCESSLIST, but has already flushed
  "if this client is already on ACCESSLIST, print error message and return"
  if other clients had READ access, pager_read_page() gets called
NON-FLUSH LOCK COMPLETED
  only now does the flush get processed
  if client on ACCESSLIST, stuff happens

client had READ access
lock_request (FLUSH)
flush occurs internally to the kernel, but the lock_request requires more processing
there's another access to the page
DATA REQUEST from flushed client for READ access - it's on ACCESSLIST, but has already flushed
  "if this client is already on ACCESSLIST, print error message and return"
  change to: if this client is already on ACCESSLIST for WRITE access, print error message and return
                XXX: what if kernel had WRITE access, but page wasn't dirty, so it didn't get returned?
             if this client is already on ACCESSLIST for READ access, it flushed and is trying to re-aquire access,
                so check to see if we've got an outstanding lock w/ flush flag set
                if so, [X] add client to WAITLIST and return (flush code will run later)
                       [ ] clear flush flag and run flush code now (more complicated but a bit faster)
                otherwise, remove client from ACCESSLIST and proceed (kernel flushed without being asked)
  if other clients had READ access, pager_read_page() gets called
lock_completed message
  only now does the flush get processed
  if client on ACCESSLIST, stuff happens


multi-page flush (pager_flush or pager_return)
lock_request sent to a client on a page where that client has no access
lock_request processed on that page
client requests access to that page

multi-page flush (pager_flush or pager_return)
lock_request sent to a client on a page where that client has no access
client requests access to that page
lock_request processed on that page



FLUSHING:

Because of the kernel's ability to proactively flush pages (unless PRECIOUS is specified),
we can't be certain that a client on a page's ACCESSLIST actually possesses that page.
All we know for sure is that a client not on ACCESSLIST does not possess the page.

Examining the Mach kernel code for m_o_lock_request, we see that the kernel might block
while looping over the pages (and unlocks the memory object while doing so), so for a
multi-page lock request, we need to consider the possibility that part of the page range
might be processed, then other kernel operations happen (including VM page operations)
before the remaining part of the page range is processed and the lock request completes.


Scenario A
- five clients have read access to page 50 and are on its ACCESSLIST (WAITLIST is empty)
- pager_flush is called on pages 0-100
- page 50 flushes on client A
- client A re-requests access to page 50, and it is supplied (ACCESSLIST and WAITLIST don't change)
- the flushes complete
- only client A now has access to page 50 and should be alone on its ACCESSLIST

Scenario A shows that ACCESSLIST, WAITLIST and a FLUSHING flag aren't enough; we have to track
which clients requested access during the flush to correctly compute ACCESSLIST at the end.


USEFUL CHANGES TO THE KERNEL:
  - m_o_data_return could specify what permissions the kernel still holds on the page.  Then
    we could easily downgrade from WRITE to READ access without waiting for a lock completed
    message, or (like the current code), never downgrading from WRITE to READ.
  - m_o_lock_completed message could specify what kind of lock was completed, avoiding the
    need to match reply ports to lock messages unless (like the current code), we just
    wait for all outstanding locks to complete before continuing with any of them.
  - libpager could use the ability to request a copy of a unmodified, un-precious page from the kernel,
    so it can serve out the page to other nodes without reading again from the disk
  - ability to notify memory manager when a non-precious page has been evicted
    (over the network, precious is heavy weight just to get notification that a page has been evicted)
  - a kernel name port common across all memory objects
    this would allow sharing pagemaps between memory objects with the same usage pattern
    (i.e, kernels A and B both map files C and D)


PRECIOUS PAGES

Flagging a page precious (so the kernel always returns it to us) isn't that big a deal
for a local client, because "returning the page" just means flipping bits in the VM tables,
but it's a lot heavier operation for a remote client, since the page will be transmitted
across the network unnecessarily.

Flagging all of our pages "precious" is tempting.  It provides us with a mechanism
for retrieving a local, in-memory page that we want to serve out over the network
without having to read it again from disk.  It also simplifies the lock completed
logic, since we don't have to worry about whether the page got returned or not
when we get the lock completed message - precious pages always get returned.


initial client goes on WAITLIST only when:
1. client has requested any access and page is PAGINGOUT, or
2. client has requested any access and another client has WRITE access
3. client has requested WRITE access (data request or unlock) and another client has any access
4. client has requested any access and is waiting for the page to page in
     READ access requested - only READ clients on ACCESSLIST
     WRITE access requested - ACCESSLIST empty

additional clients go on WAITLIST when they request almost any access
additional clients DON'T go on WAITLIST when:
1. we're sending an immediate m_o_data_error
2. they're trying to unlock and we anticipate that they'll get a flush soon by virtue of being on ACCESSLIST

Theorem: if WAITLIST is not empty, and both ACCESSLIST and WAITLIST contain only READ clients,
         then we're waiting for a page in to complete
Proof: ???
Converse: false; while waiting for a page in, we could get a request for WRITE access

Theorem: there can be at most one client on both ACCESSLIST and WAITLIST simultaneously

Theorem: if anything is on WAITLIST, we've got a flush request outstanding

Theorem: if WAITLIST is empty, we've got no flush requests outstanding

PROBLEM CONDITION:
client A has READ access
client B requests READ access
page in starts; client A on ACCESSLIST; client B goes on WAITLIST (for READ)
client A requests WRITE access via unlock request
WAITLIST is not empty, but we're not trying to flush
--- what should happen? ---
add client A to WAITLIST for WRITE
wait for page in to finish; client B gets data and moves to ACCESSLIST
now A and B are on ACCESSLIST with READ access, and A is on WAITLIST for WRITE
client B is sent a flush request, after an optional delay
once client B flushes, client A is unlocked and is sent a lock message
now A is on ACCESSLIST with WRITE access

...vs...

client A has READ access
client B requests WRITE access
flush sent to client A; client A on ACCESSLIST; client B goes on WAITLIST (for WRITE)
client A requests WRITE access via unlock request
--- what should happen? ---
nothing - client A will process the flush as an answer to its unlock request
after flush is acknowledged, page in starts
after page in finishes, client B gets data and moves from WAITLIST to ACCESSLIST

if a client is requesting unlock, then it's already on ACCESSLIST for READ
if anything is on WAITLIST for WRITE, then we're trying to flush  (right?)
if the only things on WAITLIST are for READ, then we're waiting for a page in

client A has READ access
client B requests READ access
page in starts; client A on ACCESSLIST; client B goes on WAITLIST (for READ)
client C requests WRITE access via data request
cleint C gets added to WAITLIST (for WRITE)
client A requests WRITE access via unlock request
WAITLIST is not empty, and there's a client on it waiting for WRITE, but we're not trying to flush
--- what should happen? ---
add client A to WAITLIST for WRITE
wait for page in to finish; client B gets data and moves to ACCESSLIST
now A and B are on ACCESSLIST with READ access, and C and A are on WAITLIST for WRITE
clients A and B are sent a flush request, after an optional delay
the flush requests answers A's unlock, so it comes off the WAITLIST
once clients A and B flush, client C gets the data
client A should re-request WRITE access via data request

... or ...

client A has READ access
client B requests READ access
page in starts; client A on ACCESSLIST; client B goes on WAITLIST (for READ)
client C requests WRITE access via data request
cleint C gets added to WAITLIST (for WRITE)
client A requests WRITE access via unlock request
WAITLIST is not empty, and there's a client on it waiting for WRITE, but we're not trying to flush
--- what should happen? ---
nothing - client A will soon get a flush
wait for page in to finish; client B gets data and moves to ACCESSLIST
now A and B are on ACCESSLIST with READ access, and C and A are on WAITLIST for WRITE
clients A and B are sent a flush request, after an optional delay
once clients A and B flush, client C gets the data
client A should re-request WRITE access via data request


PROBLEM

client A has READ access
client B requests WRITE access, so a flush request gets sent to client A and client B goes on WAITLIST
   ACCESSLIST: A(read)   WAITLIST: B(for write)
client A requests WRITE access (unlock request) and goes on WAITLIST
client A processes flush and requests WRITE access, so goes on WAITLIST again with a data request
client A is now on WAITLIST twice


**** TEST CODE ****

client operations:
1. request READ access
2. request WRITE access
3. request unlock READ -> WRITE
4. flush page
5. service a message from the memory manager

No access - 1 2 or 5
Read access - 3 4 or 5
Write access - 4 or 5

translator operations:
1. discard request
2. return request
3. sync request
4. complete a pending pager_read, pager_write, or pager_unlock (which one? there could be several)

1 2 3 can be synchronous or asynchronous

Test for BUG 1:
  A2 T4 (the read completes) A5 (the data_supply is processed)
    T3 T3 T3 (the syncs) A5 A5 A5 (the syncs are processed) T4 T4 T4 (the writes complete)

client_request_read_access(client, start_page, end_page)
  ASSERT: page absent in client
  sends m_o_data_request
client_request_write_access(client, start_page, end_page)
  ASSERT: page absent in client
  sends m_o_data_request
client_request_unlock(client, start_page, end_page)
  ASSERT: page present in client with read access
  sends m_o_data_unlock
client_flush(client, start_page, end_page)
  ASSERT: pages present
  if pages have WRITE access, modify data and data_return
  if pages have READ access and are precious, data_return
client_service_message(client)
  read message
  data_supply - save pointers, read/write and precious (reply if requested)
  data_lock - send messages, update pointers, reply if requested
  data_error - ??

Test for BUG 3: (one client, two pages A and B)
  AB2 T4 (the read completes) TA3 A5 (the sync is processed) TAB2 A5 (the sync is processed)
    B2 (page request) T4 (the read completes)
    (now we need to close things out and see that data got discarded)

data should always be modified if possible (whenever a client's got WRITE access)

Check:
[ ] write access to a page is exclusive to a single client
[p] pages get written back in-order (though skips are possible)
[ ] if a page is modified, nobody ever sees an earlier version of it
[ ] all data requests are properly answered with a data supply or a data error (or data unavailable)
[ ] all data unlock requests are properly answered with a data lock
[ ] if a data lock requests a reply, make sure one is received

p = partially done
